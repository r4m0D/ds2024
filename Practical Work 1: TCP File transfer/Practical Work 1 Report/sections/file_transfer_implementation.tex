\section{File transfer implementation}

\subsection{Socket Creation}

Both the client and the server start by creating a TCP socket using the \verb|socket()| function. This socket acts as an endpoint for communication.
\begin{verbatim}
    sockfd = socket(AF_INET, SOCK_STREAM, 0);
\end{verbatim}

\subsection{Connection Setup (Client) / Binding and Listening (Server)}

\subsubsection{Connection Setup (Client)}

The client uses the \verb|connect()| function to establish a connection to the server.
\begin{verbatim}
    e = connect(sockfd, (struct sockaddr*)&server_addr, sizeof(server_addr));
\end{verbatim}

\subsubsection{Binding and Listening (Server)}

The server uses \verb|bind()| to bind the socket to an IP address and port, and listen() to start listening for incoming connections.
\begin{verbatim}
    e = bind(sockfd, (struct sockaddr*)&server_addr, sizeof(server_addr));
    listen(sockfd, 10);
\end{verbatim}

\subsection{Accepting Connections (Server)}

The server accepts an incoming connection using \verb|accept()|, which returns a new socket file descriptor for the established connection.
\begin{verbatim}
    int new_sock = accept(sockfd, (struct sockaddr*)&new_addr, &addr_size);
\end{verbatim}

\subsection{File Transfer (Client)}

Before sending the file, the client opens the file using \verb|fopen()| and then sends the file's data in chunks using a while loop and the \verb|send()| function.
\begin{verbatim}
    fp = fopen(filename, "r");
    send_file(fp, sockfd);
\end{verbatim}

\subsection{File Reception and Writing (Server)}

The server receives the data in chunks using a while loop and the \verb|recv()| function, writing each chunk to the file using \verb|fprintf()|.
\begin{verbatim}
    write_file(new_sock);
\end{verbatim}

\subsection{Closing the Connection}

Finally, after the file transfer is complete, both the client close the respective socket descriptors with \verb|close()|.
